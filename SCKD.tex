\documentclass[twocolumn]{book}
\usepackage{xparse}
\usepackage{tlhPackage}
\usepackage{paralist}
\usepackage{setspace}
\parindent=0pt

\begin{document}

\title{The Semi-Comprehensive Klingon Dictionary}
\author{\small{Compiled by}\\\normalsize{Noah Bogart}}
\date{2012-07-15}
\maketitle

\tableofcontents

\chapter{\tlhe{b}}
\begin{singlespace}
\tlhw{bach}\tlhd{noun}{Shot.}\tlhd{verb}{Shoot.}\\
\tlhcf{baH}
\tlhrf{The verb used for shoot when referring to disruptors is \tlhe{bach}. Technically speaking, one shoots the energy beam from the disruptor. The general word for any energy beam (``ray'')is \tlhe{tIH}, so a disruptor's beam is \tlhe{nISwI' tIH}. Thus, the correct formation is \tlhe{nISwI' tIH bach} (``shoot the disruptor beam''). As a practical matter, however, the \tlhe{tIH} is often left out, and \tlhe{nISwI' bach} is the common way to say ``shoot a disruptor''. Similarly, \tlhe{pu' bach} is ``shoot a phaser.''}{KGT}{56}
\textit{Canon:}\tlhc{bach Do', qaH}{A lucky shot sir...}{ST3}
\tlhc{logh veQDaq bachchugh, yoH 'e' toblaHbe' SuvwI'}{Shooting space garbage is no test of a warrior's mettle.}{ST5 notes}
\tlhf{TKD}[TKW][KGT]

\tlhw{bachHa'}\tlhd{verb, slang}{Err.}[Make a mistake.]\\
\tlhcf{Qach}[muj][lughbe'][qarbe']
\textit{Commentary:} \tlhco{Note that this is obviously not a noun. I'm convinced that he made this mistake on purpose, given the definition. It sets the tone for the degree of perfection we are to expect from this work. He knows there will be errors, so why wait. Put one in the first entry.}{charghwI'}
\textit{Canon:}\tlhc{jIbachHa'pu'}{I have made a mistake (slang: ``I have mis-shot'')}{KGT}
\tlhc{bIQ ngaS HIvje'}{``The cup contains water'' - ``This idiomatic expression means ``be quick mistaken, be totally wrong''.''}{KGT 20}
\tlhpun{botch}
\tlhf{KGT}

\tlhw{bagh}\tlhd{verb}{(to) Tie.}\\
\tlhcf{muv}[rar][tay']
\tlhf{TKDa}

\tlhw{baghneQ}\tlhd{noun}{Spoon.}\\
\tlhcf{bo'Dagh}
\tlhrf{Eating is done with hands only. There is no Klingon fork or spoon. If the cook has prepared the food properly, there should be no need to use a knife either, though, from time to time, one is quite useful.}{KGT}{99}
\textit{Commentary:}\tlhco{The Klingon word for "fork" is \tlhe{puq chonnaQ}. As is well known, Klingons prefer to get food into their mouths without the aid of implements (except for such things as the bowl containing soup or the goblet containing bloodwine). Nevertheless, they have become acquainted with the eating habits of other cultures and have become aware of such things as forks. On occasion, they'll even use the implements, most commonly when partaking of a non-Klingon meal (whether on a Klingon planet or elsewhere) but sometimes when eating Klingon food, as if to add an exotic touch to the meal experience. (Not all Klingons are skilled in using forks, however, and some simply refuse to deal with them. Those who do not use them seem to be not at all troubled by eating "foreign" food using Klingon means -- that is, hands.) ... The Klingon word for "spoon" is \tlhe{baghneQ}. Even though spoons were never typically used when eating, the word appears to have been in the language for a long time, suggesting that it may once have meant something else. One theory is that it comes from \tlhe{nagh beQ} "flat stone, flat rock" and that the initial sounds of the two words, n and b, were, for some reason, transposed. This is, however, just speculation."}{st.k 5/05/98}
\tlhpun{a spoonerism}
\tlhf{st.k}

\tlhw{baH}\tlhd{verb}{Fire (torpedo, rocket, missle.)}\\
\tlhcf{bach}[laQ]
\tlhrf{The verb used for ``launch'' or ``fire'' a weapon of this type is \tlhe{baH}, and there is even a special word, \tlhe{ghuS}, meaning prepare to launch.}{KGT}{57}
\textit{Commentary:}\tlhco{This was one of the eleven original Klingon words created by James Doohan and spoken by Mark Lenard (through a set of prosthetic teeth) in the opening scene of STMP in which Marc Okrand used as the basis for \tlhe{tlhIngan Hol}. \tlhw{baH} for ``Fire!'' was clipped and, in Lenard's character's dialect, pronounced ``\tlhe{maH}'' (as mentioned in TKD). ``(Having said that, the command meaning `Fire [a torpedo]!' - which I transcribed as ``\tlhw{baH}'' but which also sounds kind of like``maH'' - must have always had that meaning, since it's there a couple of times. [The H is pronounced like the final ch in the name of the composer Bach.])''\,}{M.Ok. BBS 2/98}
\tlhco{Doohan, who knows some Russian MUST have gotten this from Russian. In Russian BAX pronounced /baH/ means BANG! as in a gun shot. I can't help but believe that this was deliberate on the part of Doohan, despite protestations by the Philistines.}{(G. Proechel) STMP}
\textit{Canon:}\tlhc{baH}{Fire!}{CK}
\tlhc{baHchu'}{He fire (the torpedo) perfectly.}{TKD}
\tlhc{yIbaH!}{Fire (the torpedo)!}{KGT}
\tlhf{TKD}[KGT]

\tlhw{baHjan}\tlhd{noun}{Launcher.}\\
\textit{Canon:}\tlhc{peng baHjan}{Torpedo launcher.}{BoP}
\tlhc{'otlh peng baHjan}{Photon torpedo launcher.}{BoP}
\tlhf{BoP}

\tlhw{baHwI'}\tlhd{noun}{Gunner.}\\
\tlhcf{matHa'}
\tlhrf{...which consists of the verb \tlhe{baH} \textit{fire (a torpedo)} plus \tlhe{-wI'} \textit{one who does}. Thus, \tlhe{baHwI'} is literally ``one who fires [a torpedo].}{TKD}{20}
\textit{Canon:}\tlhc{baHwI'pu'vam}{these gunners}{TKD}
\tlhc{baHwI', DoS yUbuS. QuQ neH. yaj'a'}{Gunner: Target engine only. Understood?}{ST3}
\tlhc{po'neS baHwI'pu'lI'}{Gunner, stand ready.}{TDP}
\tlhc{baHwI', 'ejyo'waw' DoS}{Target the starbase, gunner.}{TDP}
\tlhf{TKD}

\end{singlespace}
\end{document}
